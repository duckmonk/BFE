\documentclass{pl_template}  % 使用自定义的类文件
\usepackage{ulem}
\usepackage{soul}
\usepackage{etoolbox}  % 导入 etoolbox 包

\begin{document}

% 定义变量
\newcommand{\Petitioner}{REPLACE-PETITIONER-NAME}
\newcommand{\TypeOfPetition}{Type of Petition: REPLACE-TYPE-OF-PETITION}
\newcommand{\TypeOfPetitionShort}{REPLACE-TYPE-OF-PETITION}
\newcommand{\HeSheLow}{REPLACE-HE-SHE-LOW}
\newcommand{\HeSheUp}{REPLACE-HE-SHE-UP}
\newcommand{\HisHerLow}{REPLACE-HIS-HER-LOW}
\newcommand{\MrMs}{REPLACE-MR-MS}
\newcommand{\AttorneyName}{REPLACE-ATTORNEY-NAME}
\newcommand{\LawFirmName}{REPLACE-LAW-FIRM-NAME}
\newcommand{\LawFirmAddress}{REPLACE-LAW-FIRM-ADDRESS}
\newcommand{\LawFirmPhone}{REPLACE-LAW-FIRM-PHONE}
\newcommand{\LawFirmEmail}{REPLACE-LAW-FIRM-EMAIL}
\newcommand{\LawFirmWebsite}{REPLACE-LAW-FIRM-WEBSITE}

% 设置首页的页眉和页脚
% 设置首页的页眉和页脚
\newcommand{\firstpageheader}{
    \thispagestyle{fancy}  % 使用 fancy 样式
    \fancyhead[L]{\textbf{\LawFirmName}\\~\\~}  % 左侧页眉
    \fancyhead[R]{\LawFirmAddress\\
    Phone: \LawFirmPhone \textbar{} Email: \LawFirmEmail\\
    \LawFirmWebsite}  % 右侧页眉
    \fancyfoot[C]{\thepage}  % 页脚中央显示页码
}

% 设置其它页面的页眉和页脚
\newcommand{\otherpagesheader}{
    \fancyhead[L]{Petitioner/Beneficiary: \Petitioner \\ \TypeOfPetition}  % 左侧页眉
    \fancyhead[R]{}  % 右侧页眉为空白
    \fancyfoot[]{}   % 清除默认页脚
    \fancyfoot[C]{\thepage}  % 页脚中央显示页码
    \renewcommand{\headrulewidth}{0pt}  % 禁用页眉的横线
}

% 重定义 plain 页眉样式,使目录页使用其他页面的页眉设置
\fancypagestyle{plain}{
    \fancyhf{}  % 清空页眉和页脚
    \fancyhead[L]{Petitioner/Beneficiary: \Petitioner \\ \TypeOfPetition}  % 左侧页眉
    \fancyfoot[C]{\thepage}  % 页脚中央显示页码
    \renewcommand{\headrulewidth}{0pt}  % 禁用页眉的横线
}

% 定义书信格式结尾,右对齐
\newcommand{\letterclosing}{
    \vspace{1cm}  % 上方留出空间,确保结尾部分不会紧贴正文
    \noindent  Sincerely, \\  % 使用 \hfill 使其右对齐
    \vspace{1cm}  % 留出签名的空白
    \noindent \AttorneyName, Esq  \\  % 使用 \hfill 使其右对齐
    \noindent \LawFirmName  \\  % 使用 \hfill 使其右对齐
}

\firstpageheader


% 首页内容
\begin{center}
    \today
\end{center}

\vspace{2em}
\raggedright{  % 左对齐
VIA REPLACE-MAILING-SERVICE\\
REPLACE-RECEIVING-ADDRESS\\
\vspace{2em}
\textbf{Re: EB-2 Immigrant Petition for Permanent Residency with request for a National Interest Waiver} \\
\textbf{Petitioner/Beneficiary:} \Petitioner\\
\textbf{Type of Petition:} \TypeOfPetitionShort\\
\textbf{Classification Sought:}  Immigration and Nationality Act 203(b)(2)(B) \\
}

\vspace{3em}

\noindent Dear USCIS Officer:

\hspace*{2em} Please be advised that the undersigned counsel represents the above-referenced Petitioner/Beneficiary \textbf{\MrMs.\  \Petitioner} (“Petitioner”) in \HisHerLow \   immigration matters. This letter is respectfully submitted in support of Petitioner’s Immigrant Petition for Alien Worker (I-140).

\hspace*{2em}Petitioner is eligible for EB-2 classification as a member of the professions holding an advanced degree under section 203(b)(2) of the Immigration and Nationality Act, 8 U.S.C. §1153(b)(2) because REPLACE-EDUCATION-SUMMARY.

\hspace*{2em}Petitioner believes that \HeSheLow \ should be granted a National Interest Waiver because \HeSheLow \  can demonstrate: (1) \HisHerLow \  proposed endeavor has both substantial merit and national importance; (2) that \HeSheLow is well positioned to advance the proposed endeavor; and (3) that, on balance, it would be beneficial to the United States to waive the job offer and labor certification requirements. \textit{Matter of Dhanasar}, 26 I\&N Dec. 884 (AAO 2016)



%\vspace{2em}
%\newpage
%\firstpageheader

\newpage  % 手动分页
% 其他页面内容
\otherpagesheader  % 设置后续页面的页眉
\tableofcontents
\section{EB-2 Eligibility Statement}
\Petitioner \  qualifies for the EB-2 classification as a professional with an advanced degree. \HeSheUp \ has completed degrees in fields critical to national security and public policy: 
REPLACE-EDUCATION-LIST
These academic credentials affirm \Petitioner’s expertise and support \HisHerLow \ eligibility for the EB-2 classification.

% START INSERTION


\letterclosing  % 在此处插入书信结尾

\vspace{2em}

\section*{Exhibit List}
REPLACE-EXHIBIT-LIST
 
\end{document}
